% !TEX root = ../thesis-sample.tex

% --------- FRONT MATTER PAGES ---------------------
% Title of the thesis
\title{What's the Title of Your Dissertation?}
% capitalize significant words!

% Author name
\author{Your Name Here}

% Previous degrees
\bachelordegree{Your Bachelor's degree}
\bsdepartment{Your old department}
\bsschool{Your alma mater}
\bsgrad{Month, YYYY} % month, year

\masterdegree{M.S.}
\msdepartment{Your graduate old department}
\msschool{Your graduate alma mater}
\msgrad{Month, YYYY}  % month, year
% you can show or hide the MS degree line
\showmsdegree
% \hidemsdegree

% PhD degree commands
% Committee
\showcommitteepage % hide this page if you're doing a MS thesis
%\hidecommitteepage 


% define COMMITTEE information

% Chair must be entered separately for formatting reasons.
\chair{Professor A. Person}
\department{Your Department}
\chairtitle{Professor-Title of \insertdepartment}

\phdschool{The School of Engineering and Applied Science}

\committee{ 
% director first
\insertchair, \insertchairtitle, Dissertation Director

% remember to add a space between committee members
\vspace{\baselineskip}

% you don't need to write "The George Washington University" every time
Full Name, Title, Dissertation Director/Dissertation Co-Director/Committee Member

\vspace{\baselineskip}

Full Name, Title, Dissertation Director/Dissertation Co-Director/Committee Member

\vspace{\baselineskip}

Full Name, Title, Dissertation Director/Dissertation Co-Director/Committee Member
\vspace{\baselineskip}

% external examiner
Name of External Examiner, Title, Department, The External University, Committee Member  % include university of any external examiner! but still "committee member"
}

\phdgrad{March 8, 2020}  % Month, DD, YYYY
\defensedate{December 18, 2019}  % Month, DD, YYYY
% Year of completion for copyright page and perhaps other places
\year=2020

% Copyright page
%\copyrightholder{Someone else}

% Dedication
\dedication{ %
\emph{Include a fancy quote or dedication.}
}

% Acknowledgments
\acknowledgments{
    Here you can acknowledge all of those people who have helped you to reach this point.
    It's rare that any work is done in a vacuum and your research is no exception.
    Feel free to be grateful for all those who've aided you along your way.
}

% -----------------------------------------------------------------
% Typically only one of Preface/Foreword/Prologue would be in your thesis.
% To choose one simply delete the others and they will automatically disappear

% Preface
\preface{
    This is the preface. 
    It's another front matter page that offers additional detail into your work.
    Typically, only one (preface OR prologue OR foreword) is used. 
    You can remove the other sections by deleting them inside \texttt{tex/frontmatter.tex} or using the appropriate show or hide commands.
}

\prologue{
    This is the prologue. 
    It's another front matter page that offers additional detail into your work.
    Typically, only one (preface OR prologue OR foreword) is used. 
    You can remove the other sections by deleting them inside \texttt{tex/frontmatter.tex} or using the appropriate show or hide commands.
}

\foreword[2]{
    This is the foreword. 
    It's another front matter page that offers additional detail into your work.
    Typically, only one (preface OR prologue OR foreword) is used. 
    You can remove the other sections by deleting them inside \texttt{tex/frontmatter.tex} or using the appropriate show or hide commands.
}
% ----------------------------------------------------------------------

% commands to show or hide front matter pages

\showcopyright
\showabstract
\showcommitteepage
\showdedication
\showacknowledgments
\showpreface
\hideprologue
\hideforeword

% ------------ TABLE OF CONTENTS ----------------------
% add 'chapter' preceding the snum
\renewcommand{\cftchappresnum}{\chaptername\space}
\renewcommand{\cftchapaftersnum}{:}
\setlength{\cftchapnumwidth}{\widthof{Appendix AAA~ }}
% change this to 'appendix' by hacking the \appendix command
\makeatletter
\g@addto@macro\appendix{%
  \addtocontents{toc}{%
    \protect\renewcommand{\protect\cftchappresnum}{\appendixname\space}%
  }%
}
\makeatother

% Commands to hide or show lists of figures, tables, etc.
\showlistoffigures
\showlistoftables
\hidenomenclature

% --------- ACRONYMS and SYMBOLS ------------------------------
% TODO Deprecate the entire acronym package and switch to glossaries

% You can either use the acronym or glossaries package (both work)
% Definition of any abbreviations used.
\abbreviations{
    \acro{CRTBP}{Circular Restricted Three Body Problem}
    \acro{NSA}{National Security Agency}
    \acro{SSME}{Space Shuttle Main Engine}
    \acro{HIDEME}{Hide the pages for One of These Abbreviation Systems; this is from abbreviations}
}
% call an abbreviation using \ac{abbrev}

% symbols and acronyms only show up when used in the text
\symbols{
    \acro{J}{Moment of Inertia}
}       

% if you want acronym (simpler) then change these to show
\showlistofabbreviations
\showlistofsymbols

% if you want glossaries (more powerful) then leave above as hide
% GLOSSARIES package options - automatically turns off front pages from acronym package

% acronymns and symbols are basically the same, but there are two provided 
% locations where they can show up
\setabbreviationstyle[acronym]{long-short}
\setabbreviationstyle[abbreviation]{long-short}
\makeglossaries
% you can hide/show the glossaries page
\showglossarieslistofabbreviations
\showglossarieslistofsymbols
\showglossariesglossaryofterms

% acronyms defined in glossaries
\newabbreviation{crtbp}{CRTBP}{Circular Restricted Three Body Problem}
\newabbreviation{lidar}{LIDAR}{Light Detection and Ranging}
\newabbreviation{hideme}{HIDEME}{Hide One of These Abbreviation Systems; this is from Glossaries}

% defining abbreviations like this allows for autocompletion
\newglossaryentry{filo}{
    name={FILO},
    type=\glsxtrabbrvtype,
    description={first in last out},
    first={first in last out (FILO)}
}

% glossary entries
\newglossaryentry{linux}{
    name=Linux,
    description={is a generic term referring to the family of Unix-like computer operating systems that use the Linux kernel},
    plural=Linuces
}

\newglossaryentry{matrix}{
    name={matrix},
    plural={matrices},
    description={rectangular array of quanttities}
}

% symbols
\newglossaryentry{M}{
    type=symbols,
    name={\ensuremath{M}},
    sort=M,
    description={a \gls{matrix}}
}

\newglossaryentry{F}{
    type=symbols,
    name={\ensuremath{F}},
    sort=F,
    description={External Force}
}

% Some abstract text
\abstract{
This is the abstract. 
It contains some random text from the \texttt{lipsum} package. 
You may safely remove the \texttt{lipsum} package once you write your thesis.

\lipsum[1]
}
